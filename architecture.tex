L'architecture a été réalisé sous la forme d'un enchainement de vues (fenêtres de l'application correspondant à ce que l'utilisateur final pourra voir) très simples (aucune notion de charte graphique, aucun style particulier, pas d'image, ...) permettant de savoir que devait contenir chaque vue (bouton, texte, etc) ainsi que les liens entre ces différentes vue (\og{} si j'appuie sur ce bouton, je vais où ?\fg{}).
\newline
L'architecture de l'application est représenté sur le diagramme~\ref{architecture_all}. La légende donne l'explication des différents éléments graphiques, les flèches représentent quand à elles un changement de vue.

Comme le montre le diagramme, l'application commencera sur un simple menu avec les choix suivant:
\begin{description}
\item[Jouer:] lance le choix de l'exercice
\item[Chercher des jeux:] amène au catalogue d'applications (connexion à internet nécessaire exclusivement pour cette partie). Ce catalogue permettra de lister les différents \plugins{} disponibles et aidera ainsi le téléchargement de nouveaux exercices)
\item[Mon carnet:] petit outil statistique permettant de voir la prograssion du joueur dans les différents exercices
\item[Quitter:] quitte simplement l'application
\end{description}

\begin{figure}[htp]
	\centering
	\includegraphics[width=14cm]{\dossierimages architecture_all}
	\caption{Architecture}
	\label{architecture_all}
\end{figure}De manière générale, la vue de démarrage d’un exercice est une liste des séquences de questions proposées. Cette vue doit être normalisée. Les vues des questions sont en revanche presque totalement libres (le design graphique doit être standardisé). Un plugin est basé sur la proposition de Stéphane Legrand. C’est le plugin qui “dira” que l’exercice est fini en donnant le score (à l’aide d’outils fournis dans une classe mère “ExerciceGenerique”).

Vous pouvez remarquer que les exercices ne sont pas fortement détaillés sur ce diagramme, nous avons décidé de séparer l'architecture en deux diagramme. Le premier dont on vient de parler et le second décrivant un peu plus en détail un exercice (diagramme~\ref{architecture_exercice}).

\begin{figure}[htp]
	\centering
	\includegraphics[width=8cm]{\dossierimages architecture_exercice}
	\caption{Architecture d'un exercice}
	\label{architecture_exercice}
\end{figure}

De manière générale, la vue de démarrage d’un exercice est une liste des séquences de questions proposées. Cette vue sera normalisée est gérré par l'application. Les vues des questions seront en revanche presque totalement libres (le design graphique doit être standardisé). Les exercices étant prévus sous forme de \og{}Plugins\fg{} additionnels, ils pourront \^{e}tre très différents les uns des autres. Voir la partie sur les Plugins (partie~\ref{partie_plugin} page~\pageref{partie_plugin}) pour plus de détails.

% TODO rajouter la reférence à l'annexe (ainsi que l'annexe)
Le document complet décrivant l'architecture est disponible en annexe %(annexe~\ref{} page~\pageref{})