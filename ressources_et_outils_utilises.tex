\section{Documentation}
\subsection{\googleDrive}
Nous utilisons cette plate-forme collaborative, pour une meilleures gestions et partages des fichiers du groupe de travail. Nous privilégions l'utilisation des formats de document \google pour une meilleures harmonisation et compatibilité sur le cloud.

Raisons techniques de ce choix :
\begin{itemize}
\item Un seul point de stockage pour le groupe
\item Suite bureautique à disposition
\item Stockage de fichiers, peu importe le type de fichier
\item Plate-forme collaborative (édition simultanée, commentaires, ...)
\end{itemize}

\subsection{Latex}
Nous utilisons \LaTeX{} car \googleDocuments{} ne permet pas de réaliser une bonne mise en page pour un dossier, gr\^{a}ce à \LaTeX{} nous pouvons facilement créer des tables de matières ou des liens hypertexte par exemple, ce qui n'est pas aisé sur \googleDocuments{}.

Nous utilisons \LaTeX{} juste pour le rapport de mi-parcours et le rapport final.

%% ************************************************** %
\section{Développement}
%Pour développer des applications \android{}, \google{} préconise l'utilisation d'\eclipse{} sur \ubuntu{}.
\begin{description}
\item[\os{}:] Le développement se fait sur différentes distributions \linux{}
\item[\ide{} et language:] Pour le développement, nous utilisons le SDK d'\android{} couplé avec l'\ide{} \eclipse{}. Le langage utilisé est \java{}
\item[Gestionnaire de depôts:] Ce projet étant réalisé en équipe qui ne se voit pas régulièrement. Il a été décidé dans un premier temps (par \responsableProjet{}) d'utiliser un gestionnaire de sources. Nous utilisons \github{} à la fois pour les sources du projet mais aussi (dans un second projet \github{}) les sources de notre rapport
\end{description}
%%% ************************************************** %
\section{Android}
Version min du SDK :
\begin{description}
\item[API : ] 12
\item[Version : ] 3.1 
\item[Codename : ] Honeycomb 
\end{description}

Version pour développement du SDK :
\begin{description}
\item[API : ] 17
\item[Version : ] 4.2 
\item[Codename : ] Jelly Bean
\end{description}