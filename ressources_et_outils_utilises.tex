\section{Documentation}
\subsection{\googleDrive}
Il a été décidé lors de notre première réunion (voir annexe \ref{reunion1} page \pageref{reunion1}) que les documents de travails (exemple: résumé de réunion, glossaire, ...) sont a stocker dans un repertoire de \googleDrive{}. Les document de type \og{}\google{}\fg{} sont à privilégier.
\subsection{Latex}
Nous nous sommes servi du langage \LaTeX{} uniquement pour la rédaction de ce rapport et pour notre présentation.
%% ************************************************** %
\section{Développement}
%Pour développer des applications \android{}, \google{} préconise l'utilisation d'\eclipse{} sur \ubuntu{}.
\begin{description}
\item[\os{}: ] Le développement se fait sur différentes distributions \linux{}
\item[\ide{} et language: ] Pour le développement, nous utilisons le SDK de \android{} couplé avec l'\ide{} \eclipse{}. Le language utilisé est \java{}
\item[Gestionnaire de depôts: ] Ce projet étant réalisé en équipe qui ne se voit pas régulièrement. Il a été décidé dans un premier temps (par \responsableProjet{}) d'utiliser un gestionnaire de source. Nous utilisons \github{} à la fois pour les sources du projet mais aussi (dans un second projet \github{}) les sources de notre rapport
\end{description}
%%% ************************************************** %
\section{Android}
Version min du SDK :
\begin{description}
\item[API : ] 12
\item[Version : ] 3.1 
\item[Codename : ] Honeycomb 
\end{description}

Version pour développement du SDK :
\begin{description}
\item[API : ] 17
\item[Version : ] 4.2 
\item[Codename : ] Jelly Bean
\end{description}