\subsubsection{Les plugins}
Au niveau fichier, un plugin et un fichier \og{}zip\fg{} contenant le \jar{} (archive de classes java) de l'exercice ainsi que les ressources necessaire au bon fonctionnement de l'exercice placé dans 2 sous-dossier: \og{}drawable\fg{} et \og{}sounds\fg{}. Chaque plugin est donc constitué de cette façon.

Le nom du plugin est unique, c'est en fait une concaténation d'informations: le niveau de l'exercice, son thème ainsi que son nom. Par exemple \og{}m\_francais\_lelales\fg{} signifie que ce plugin correspond à l'exercice \og{}lelales\fg{}, qui est un exercice de français de niveau maternelle.

\subsubsection{Stockage des plugins}
\label{importation_plugin-stockage}
Tous les plugins seront mis à disposition dans un même dossier sur un serveur distant. Leur réferencement a lieu par le biai d'un fichier JSON. Ce fichier contient la liste trié (par niveau puis thème) de tous les plugins téléchargables.

Au niveau de l'application, les plugins téléchargés par l'utilisateur sont stockés localement sur l'appareil. Leur référencement ce fait par la base de données locale (voir la partie~\ref{base_de_donnees} pour plus de précision sur la base de données).

\subsubsection{Test accessibilité internet}
La tablette est capable de tester si l'accès internet est actif ou non sur la tablette à tout moment.

\subsubsection{Synchronisation des listes}
Une synchronisation de la liste des plugins disponibles sur le serveur distant se fait à chaque démarage de l'application si une connexion internet est disponible (actuellement sans demande de permition à l'utilisateur).
Cette synchronisation se fait de la manière suivante:
\begin{itemize}
    \item telechargement du fichier JSON
    \item lecture (avec paring) du fichier
    \item comparaison/synchronisation avec la \bdd{} locale:
    \begin{itemize}
        \item ajout de nouveaux plugins disponible
        \item actualisation du niveau de version disponible
        \item plugin devennu indisponible non géré (on considère que l'on ne retirera jamais d'exercice)
    \end{itemize}
\end{itemize}
%Téléchargement de la liste des plugins disponible: la liste des plugins disponibles est totalement géré (téléchargement et mise-à-jour), mais les niveaux et les thèmes sont encore insérés manuellement par l'application principale. 
%Cette mise-à-jour ce fait par le téléchargement du fichier JSON (voir partie~\ref{importation_plugin-stockage}), ce fichier est lu entièrement puis comparé avec la base de données locale afin de faire les ajouts/modifications nécessaires. 
Soulignons bien que cette procédure ne télécharge aucun \plugin{} et n'en mets aucun à jour. Elle actualise simplement la \bdd{} locale permettant ainsi de connaitre la liste totale de plugins disponible ainsi que leur numéro de version actuelle.

\subsubsection{Téléchargement}
\label{importation_plugin-telechargement}
Une fonction de téléchargement d'un plugin est disponible dans l'API mais n'est encore appelé à aucun endroit. Cette fonction télécharge par paquet l'archive du \plugin{} par rapport à son nom (unique) de \plugin{}.

\subsubsection{Amélioration prévus}
Il y a quatre principales améliorations à apporter à cette partie:
\begin{itemize}
    \item synchronisation totale de la liste des \plugin{}:
    \newline Comme dit précedament, seul les plugin sont mis-à-jour. La liste des niveaux et des thèmes et encore ajouter de manière manuelle dans l'applcation principale. La synchronisation doit donc toucher aussi ces deux niveaux (la \bdd{} locale sera complétement vide au premier lancement de l'application)
    \item téléchargement d'un \plugin{}:
    \newline La méthode le permettant est disponible mais non-utilisé.
    \item avertissement pour l'utilisateur:
    \newline Une demande de permission doit être faite à l'utilisateur pour toutes manipulations demandant un accès internet (en prévennant pour les coûts suplémentaires dû à la taille des plugins, etc)
    \item l'adresse du serveur distant:
    \newline Actuellement stocké en brut dans le code, elle devra être géré par rapport aux préférences
\end{itemize}
