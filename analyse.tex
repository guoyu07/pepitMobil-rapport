\paragraph{}Après le lancement du projet, il était nécessaire de faire un état des lieux sur \pepitSite{}. Nous avons donc fait une analyse portant sur les exercices se trouvant sur le site web.

La structure des données est la suivante : Niveau -> Thème -> Exercice -> Module

\begin{description}
\item[Niveaux : ] Maternelles, Niveau 1, Niveau 2, Niveau 3, Niveau 4, Niveau 5, Niveau 6, Conjugaison, Table de multiplication, Enseignement spécial, Pour tous et Secondaire.
\item[Thèmes : ] Français, Mathématiques et Divers.
\end{description}

\paragraph{}L'application web utilise de nombreuses images, ce qui pose un problème sous \android, l'application sera de ce fait plus lourde. Ainsi que certains exercices utilisent des animations, ce qui demande une certaine ma\^{i}trise du développement mobile.
\paragraph{}L'utilisation du clavier revient souvent, mais ceci n'est pas un problème pour l'application mobile.