Comme dit précédemment, la partie sur les plugins est à la charge de \etudiantSL{}. Nous ne comptons pas détailler son travail mais nous allons en parler car la compréhension de ce point est nécessaire pour la vue d'ensemble du projet.

%Il est important de savoir que l'équipe de \pepit{} aimerait pouvoir rajouter des exercices facilement sans dépendre de nous par la suite et sans avoir besoin d'apprendre à faire des programmes \android{}.\newline
Le site \pepit{} contient déjà un bon nombre d'exercices, et ce nombre est fortement succeptible d'augmenter, de plus les exercices sont parfois modifiés (changement des images par exemple). L'equipe de \pepit{} aimerait que les utilisateurs puissent obtenir leurs nouveaux exercices facilement et ce, sans passer par \market{}. \market{} obligerait l'utilisateur soit à avoir tous les exercices (or, on veut que l'utilisateur puisse choisir ses exercices) soit à télécharger une application par exercice (ce qui noierait \market{} et serait très lourd pour l'utilisateur ainsi que pour nous).
\newline
Nous sommes donc partis sur un système de \plugins{}. Le système voulu, du point de vue de l'utilisateur, serait le suivant:
\begin{enumerate}
	\item l'utilisateur télécharge l'application sur \market{} (cette application contient une petite série d'exercice permettant à l'utilisateur de se faire de suite une bonne idée des possibilité du logiciel).
	\item l'utilisateur essaye l'application (fait certains nombre d'exercices, constate ses scores, etc)
	\item l'utilisateur cherche de nouveaux exercices (il tombe sur un catalogue lui permettant de trouver/télécharger facilement de nouveaux exercices en fonction de ses envies
	\item l'utilisateur essaye ses nouveaux exercices
	\item etc
\end{enumerate}
Tout doit bien évidemment être le plus simple possible pour l'utilisateur. Mais pour nous, lorsque l'utilisateur télécharge de nouveaux exercices (étape 3), il télécharge un plugin par nouvel exercice qui sera greffé à l'application. 

Au niveau technique, \etudiantSL{} a trouvé plusieurs possibilités pour la réalisation d'un tel système mais nous ne sommes pas encore sûrs de laquelle sera déployée (nous allons les tester plus en profondeur après avoir réalisé quelques exercices afin d'avoir de la matière plus significative).
