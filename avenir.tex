Le développement de l'application ne s'arrêtera pas avec notre projet, l'ambition du groupe est que celui-ci continue.


\subsection{Le projet}
Dans les mois futurs, l'application sera utilisée dans une école pour un test grandeur nature. 
Cela permettra d'avoir un grand nombre de retours avec les différents points de vues des utilisateurs ciblés (les enfants ainsi que les professeurs). 
Après celui-ci, nous aurons un retour concret sur les points qu'il nous reste à améliorer.


L'équipe de projet est à la recherche de contributeurs. Les contributions pouvant être de différentes natures, principalement:
\begin{itemize}
    \item évolution de l'application maître
    \item ajout de contenus
    \item site vitrine
\end{itemize}

\subsubsection{\'{E}volution de l'appplication maître}
L'application maître a besoin encore d'un certain nombre d'ajout dont nous avons parlé précédemment, par exemple:
\begin{itemize}
    \item amélioration du système d'importation d'un plugin
    \item lancement d'un plugin téléchargé (actuellement bogué)
\end{itemize}

\subsubsection{Ajout de contenus}
Concernant l'ajout de contenus, il s'agit principalement de la réécriture des exercices dans leur version \android{} (pour commencer car leur nombre est encore très limité).

\subsubsection{Site vitrine}
Le développement d'un site vitrine permettant d'ajouter/supprimer/consulter les exercices disponibles sur le serveur est une évolution très interessante dont l'équipe avait parlé mais qui n'a pas encore été réalisé par manque de temps. Ce développement est donc à séparer des autres \og{}types\fg{} de contributions.

Cette vitrine pourrait permettre par exemple de:
\begin{itemize}
    \item gérer automatiquement le fichier JSON (ajout/suppression/modification d'un exercice)
    \item faciliter l'importation des exercices aux différents contributeurs
    \item faciliter la validation des exercices proposés (complétement manuelle à l'heure actuelle)
    \item avoir un visuel du panel d'exercices disponibles sans même avoir l'application
\end{itemize}


\subsection{L'équipe de développement}
Nous (Romain et Jonathan) voulons continuer à contribuer dans ce projet (en fonction de nos emploies du temps respectifs). Cette motivation est due à un int\^{e}ret pour les technologies mobiles telles que \android{} mais aussi à notre curiosité (nous souhaitons voir jusqu'où peut aller l'application \pepitMobil{}).
