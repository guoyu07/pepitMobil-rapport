\subsection{Techniques}
Le développemment sous \android{} n'a pas été la grosse difficulté de ce projet, vu que nous le connaissions déjà bien. La plus grosse difficulté technique était la conception d'une application \android{}, qui puisse télécharger et utiliser différents \plugins{} externes très divers (comme dit précédemment, les exercices sont très loin d'être facilement généralisables). Une longue étude a dû être mise en place pour ce système. 

\subsection{Gestion de projet}
Le projet utilise Github comme plateforme de gestion de projet, ce qui nous a permis d'avoir un support pratique et fiable.
Avec la mise en place de \sprint s d'une durée d'un mois. Ce système de \sprint{} nous a motivé à travailler tout au long de l'année. Cette méthode de gestion de projet a été très efficace.


\subsection{Application}
Voici des impressions écran de la version actuelle de l'application.
\begin{figure}[H]
   	\begin{minipage}[c]{.46\linewidth}
		\includegraphics[width=7cm]{\dossierimages home} 
		\caption{Pepit.be - Page d'accueil}
		\label{Pepit.be - Page d'accueil}
   	\end{minipage} \hfill
  	\begin{minipage}[c]{.46\linewidth}
      	\includegraphics[width=7cm]{\dossierimages add}
     	\caption{Pepit.be - Ajout d'un profil}
		\label{Pepit.be - Ajout d'un profil}
   	\end{minipage}
\end{figure}
\begin{figure}[H]
   	\begin{minipage}[c]{.46\linewidth}
		\includegraphics[width=7cm]{\dossierimages change} 
		\caption{Pepit.be - Changement de profil}
		\label{Pepit.be - Changement de profil}
   	\end{minipage} \hfill
  	\begin{minipage}[c]{.46\linewidth}
      	\includegraphics[width=7cm]{\dossierimages game_2}
     	\caption{Pepit.be - Exercice}
		\label{Pepit.be - Exercice}
   	\end{minipage}
\end{figure}
