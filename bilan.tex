\subsection{Techniques}
Ce ne f\^{u}t pas une découverte pour nous de développer sous Android. La plus grosse difficulté technique fût la conception d'une application Android, qui puisse télécharger un plugin. Une longue étude à du être mis en place pour ce système.

\subsection{Gestion de projet}
Le projet utilise Github comme plâteforme de gestion de projet, ce qui nous a permis d'avoir un support pratique et fiable. Avec la mise en place de sprints d'une durée d'un mois, un rythme f\^{u}t gradé durant le projet. Ce f\^{u}t une méthode de gestion de projet très efficace.

\subsection{Application}
Voici des impressions écran de l'application en sa version actuelle.
\begin{figure}[H]
   	\begin{minipage}[c]{.46\linewidth}
		\includegraphics[width=7cm]{\dossierimages home} 
		\caption{Pepit.be - Page d'accueil}
		\label{Pepit.be - Page d'accueil}
   	\end{minipage} \hfill
  	\begin{minipage}[c]{.46\linewidth}
      	\includegraphics[width=7cm]{\dossierimages add}
     	\caption{Pepit.be - Ajout d'un profil}
		\label{Pepit.be - Ajout d'un profil}
   	\end{minipage}
\end{figure}
\begin{figure}[H]
   	\begin{minipage}[c]{.46\linewidth}
		\includegraphics[width=7cm]{\dossierimages change} 
		\caption{Pepit.be - Changement de profil}
		\label{Pepit.be - Changement de profil}
   	\end{minipage} \hfill
  	\begin{minipage}[c]{.46\linewidth}
      	\includegraphics[width=7cm]{\dossierimages game_2}
     	\caption{Pepit.be - Exercice}
		\label{Pepit.be - Exercice}
   	\end{minipage}
\end{figure}